\section{Рекурсивные функции и отношения}

Рассмотрим примитивы, из которых будем собирать выражения:

\begin{enumerate}
\item $Z: N \rightarrow N$, $Z(x) = 0$
\item $N: N \rightarrow N$, $N(x) = x'$
\item Проекция. $U^n_i: N^n \rightarrow N$, $U^n_i (x_1, ... x_n) = x_i$
\item Подстановка. Если $f: N^n \rightarrow N$ и $g_1, ... g_n: N^m \rightarrow N$, 
  то $S\langle{}f,g_1,...g_n\rangle: N^m \rightarrow N$.
При этом $S\langle{}f,g_1,...g_n\rangle (x_1,...x_m) = f(g_1(x_1,...x_m), ... g_n(x_1,...x_m))$
\item Примитивная рекурсия. Если $f: N^n \rightarrow N$ и $g: N^{n+2} \rightarrow N$, то
  $R\langle{}f,g\rangle: N^{n+1} \rightarrow N$, при этом
  $$R\langle{}f,g\rangle (x_1,...x_n,y) = \left\{\begin{array}{ll}
    f(x_1,...x_n) & , y = 0\\
    g(x_1,...x_n,y-1,R\langle{}f,g\rangle(x_1,...x_n,y-1)) &, y > 0
  \end{array}\right.$$
\item Минимизация. Если $f: N^{n+1} \rightarrow N$, то $\mu \langle{}f\rangle: N^n \rightarrow N$, при этом
  $\mu \langle{}f\rangle (x_1,...x_n)$ --- такое минимальное число $y$, что $f(x_1,...x_n,y) = 0$.
  Если такого $y$ нет, результат данного примитива неопределен.
\end{enumerate}

Если некоторая функция $N^n \rightarrow N$ может быть задана с помощью данных примитивов, 
то она называется рекурсивной. Если некоторую функцию можно собрать исключительно из первых 
5 примитивов (то есть без использования операции минимизации), то такая функция называется 
примитивно-рекурсивной. 

\begin{theorem}Следующие функции являются примитивно-рекурсивными:
сложение, умножение, ограниченное вычитание 
(которое равно 0, если результат вычитания отрицателен),
целочисленное деление, остаток от деления, проверка значения на
простоту.
\end{theorem}

\begin{proof}Упражнение.\end{proof}

\begin{definition}
Если $R$ --- $n$-местное отношение, то его \emph{характеристической функцией}
мы назовем функцию 

$$C_R(x_1,\dots x_n) = \left\{\begin{array}{rl}
    1, & \mbox{если $(x_1,\dots x_n) \in R$}\\
    0, & \mbox{если $(x_1,\dots x_n) \notin R$}
\end{array}\right.$$
\end{definition}

\begin{definition}
Рекурсивным отношением называется отношение, характеристическая функция
которого рекурсивна.
\end{definition}

\subsection{Рекурсивные, но не примитивно-рекурсивные функции}

Интересен вопрос --- если столь сложные функции, как вычисление простых
чисел, являются примитивно-рекурсивными, то не достаточно ли их будет
для вычисления любой функции? А если нет, то каков пример функции, не 
являющейся примитивно-рекурсивной.

Классическим примером рекурсивной, но не примитивно-рекурсивной, функции 
является функция Аккермана.
                    
\begin{definition}Функцией \emph{Аккермана} мы назовем так определенную 
функцию:

$$A(m,n) = \left\{\begin{array}{rl}
   n+1, & \mbox{если $m = 0$}\\
   A(m-1,1), & \mbox{если $m > 0, n = 0$}\\
   A(m-1,A(m,n-1)), & \mbox{если $m > 0, n > 0$}
\end{array}\right.$$

Также нам будет удобно другое обозначение: $\alpha_m(n) = A(m,n)$. 
Данное обозначение позволяет рассматривать функцию Аккермана как семейство
функций от одной переменной.
\end{definition}

\begin{lemma}$\alpha_{m+1}(n) = \alpha^{n+1}_m(1)$.
\end{lemma}
\begin{proof}
Докажем индукцией по $n$. 

База индукции: $\alpha_m(0) = A(m,0) = A(m-1,1) = \alpha^{0+1}_{m-1}(1)$.

Переход: пусть $\alpha_{m+1}(n) = \alpha^{n+1}_m(1)$. Тогда 
$\alpha_{m+1}(n+1) = A(m+1,n+1) = A(m,A(m+1,n)) = \alpha_m(\alpha^{n+1}_m(1)) = 
\alpha^{n+2}_m(1)$
\end{proof}

\begin{lemma}Для функции Аккермана справедливы следующие свойства:
\begin{enumerate}
\item Если $m_1 < m_2$, то $A(m_1,n) < A(m_2,n)$
\item Если $n_1 < n_2$, то $A(m,n_1) < A(m,n_2)$
\item $\alpha_1(n) = n+2$
\item $\alpha_2(n) = 2n+3$
\item $\alpha_{m+2}(n) > \alpha^{n+2}_m(n)$
\end{enumerate}
\end{lemma}

\begin{proof}
Первые четыре свойства достаточно легко показать по индукции. Покажем последнее.

Согласно определению функции Аккермана, и предыдущим пунктам леммы,
справедлива цепочка равенств и неравенств: 
$\alpha_{m+2}(n) = \alpha_{m+1}^{n+1}(1) = \alpha_{m+1}(\alpha_{m+1}^n(1)) =
\alpha_m^{\alpha_{m+1}^n(1)+1}(1) = \alpha_m^{\alpha_{m+2}(n-1)+1}(1)
\ge \alpha_m^{\alpha_2(n-1)+1}(1) = \alpha_m^{2(n-1)+3+1}(1) =
 \alpha_m^{n+n+2}(1) = \alpha_m^{n+2} (\alpha_m^{n}(1))$
%База индукции: $m = 0$. Пусть $inc(x)=x+1$. По определению мы знаем,
%что $\alpha_0(x)=inc(x)$.
%Тогда: $\alpha_2(n) = 2n+3$, а $\alpha^{n+1}_0(n) = inc^{n+1}(n) = n+n+1 = 2n+1$. 
%
Поскольку $\alpha_m^{n}(1) \ge \alpha_0^{n}(1) = inc^n(1) = n+1$, то
$\alpha_m^{n}(1) > n$, то есть $\alpha_m^{n+2} (\alpha_m^{n}(1)) > \alpha^{n+2}_m(n)$.

%Переход: пусть мы знаем, что для некоторго $m > 0$ верно, что 
%$\alpha_{m+2}(n) \ge \alpha^{n+2}_m(n)$. 
%Покажем, что $\alpha_{m+3}(n) \ge \alpha^{n+2}_{m+1}(n)$.
%
%Заметим: $\alpha_{m+3}(n) = \alpha_{m+2}(\alpha_{m+3}(n-1)) \ge
%\alpha^{\alpha_{m+3}(n-1)+n+2}_m(n)$.
%С учетом того, что $\alpha_{m+3}(n) \ge \alpha_2(n) = 2n+3$, получим
%$\alpha_{m+3}(n-1)+2 \ge 2(n-1)+3+2 \ge 2n+3$.
%А раз так, то $\alpha^{\alpha_{m+3}(n-1)+2}_m(n) \ge 
%\alpha^{2n+3}_m(n) \ge \alpha^{2n+3}_m(1) = \alpha^{n+2}_m(\alpha^{n+1}_m(1)) =
%\alpha^{n+2}_m(alpha_{m+1}(n))


%$\alpha^{\alpha_{m+3}(n-1)}_{m+1}(1) \ge \alpha^{\alpha_{m+3}(n-1)}_{m-1}(1)
%\ge 
%A^{A(R+2,x-1)+1}_{R}(1) \ge \alpha^{A(R+2,x-1)+1}_{N}(1)
%\ge \alpha^{A(2,x-1)+1}_{N}(1)$.
%
%С учетом того, что $A(2,n) = 2n+3$, продолжим:
%$\alpha^{A(2,x-1)+1}_{N}(1) = \alpha^{2x+2}_{N}(1) \ge
%\alpha^{2x+1}_{N}(1)$.

%Заметим, что $\alpha^{2x+1}_{N}(1) \ge \alpha^{x+2}_{N}(x) = \alpha_{N+1}(x)$,
%так как $\alpha_k(x) > x$ - и мы можем, применив функцию
%$x-1$ раз, увеличить $1$ на $x-1$, получив $x$. 
%Значит, $A(R+2,x) \ge \alpha_{N+1}(x)$.

\end{proof}


Введем обозначение: если задан некоторый набор значений
$x_1, \dots x_n$, то для упрощения записи будем писать $\overrightarrow{x}$
вместо него. Например, если задана некоторая функция $f: N^n \rightarrow N$,
то будем писать $f(\overrightarrow{x})$ вместо $f(x_1,\dots,x_n)$.
Запись же $g(\overrightarrow{x},y)$ означает, что функция $g$ применяется к
$n+1$ аргументу: $g(x_1, \dots, x_n, y)$.

\begin{theorem}
Функция Аккермана растет быстрее любой примитивно-рекурсивной функции. Точнее,
какова бы ни была примитивно-рекурсивная фукнция $p: N^n \rightarrow N$, 
мы можем подобрать такую константу $K$, что 
$p(\overrightarrow{x}) \le \alpha_K(\\max(\overrightarrow{x}))$ 
при любом $x$. 
\end{theorem}

\begin{proof}
Докажем индукцией по структуре формулы, показывающей примитивную 
рекурсивность $p$.
Базой будут являться примитивы Z,N и U, для которых утверждение очевидно.
Покажем переход для примитивов S и R.

Рассмотрим примитив $S$. Пусть $p = S\langle f,g_1,\dots g_m \rangle$,
где $f, g_1, \dots g_m$ --- это некоторые примитивно-рекурсивные функции, 
для которых по предположению индукции уже найдены требуемые константы 
$K_0, K_1 \dots K_m$.
Тогда возьмем $K = \max (K_0 \dots K_m)+2$, и покажем, что это значение нас
устроит.
В самом деле: 
$$p(\overrightarrow{x}) = f(g_1(\overrightarrow{x}), \dots g_m(\overrightarrow{x}))
\le \alpha_{K_0}(\max (g_1(\overrightarrow{x}), \dots g_m(\overrightarrow{x})))
\le \alpha_{K-2}(\max (g_1(\overrightarrow{x}), \dots g_m(\overrightarrow{x})))$$
В свою очередь, каждый $g_i$ мы можем оценить через соответствующий $\alpha_{K_i}$:
$$g_i(\overrightarrow{x}) \le \alpha_{K_i}(\max(\overrightarrow{x})) \le \alpha_{K-2}(\max(\overrightarrow{x}))$$
из чего получается, что 
$$\alpha_{K-2}(\max (g_1(\overrightarrow{x}), \dots g_m(\overrightarrow{x}))) \le 
\alpha_{K-2}(\alpha_{K-2}(\max(\overrightarrow{x}))) \le \alpha_{K-2}^{\max(\overrightarrow{x})+2}(\max(\overrightarrow{x}))$$
Применив свойство из предыдущей леммы, мы можем заключить требуемое:
$$\alpha_{K-2}^{\max(\overrightarrow{x})+2}(\max(\overrightarrow{x})) < \alpha_K(\max(\overrightarrow{x}))$$

Теперь рассмотрим примитив $R$: $p = R\langle f,g \rangle$,
причём $f: N^{n-1} \rightarrow N$, и $g: N^{n+1} \rightarrow N$,
и для них уже найдены константы $K_1$ и $K_2$, соответственно.
Тогда возьмем $K = \max(K_1, K_2)+2$.

Для доказательства утверждения покажем чуть более сильное утверждение:
$$p(\overrightarrow{x},y) \le \alpha_{K-2}^{y+1}(\max(\overrightarrow{x},y))$$

Этого достаточно, поскольку 
$$\alpha_{K-2}^{y+1}(\max(\overrightarrow{x},y)) \le
\alpha_{K-2}^{\max(\overrightarrow{x},y)+1}(\max(\overrightarrow{x},y)) \le$$
$$\le \alpha_{K-2}^{\max(\overrightarrow{x},y)+2}(\max(\overrightarrow{x},y)) <
\alpha_K(\max(\overrightarrow{x},y))$$

Покажем требуемое индукцией по $y$.
База: $y = 0$. Тогда: $$p(\overrightarrow{x},0) = f(\overrightarrow{x}) \le 
\alpha_{K_1}(\max(\overrightarrow{x})) = \alpha_{K_1}(\max(\overrightarrow{x},0)) \le 
\alpha_{K-2}(\max(\overrightarrow{x},0))$$

Переход: пусть $p(\overrightarrow{x},y) \le \alpha^{y+1}_{K-2}(\max(\overrightarrow{x},y))$.
Тогда $$p(\overrightarrow{x},y+1) = g(\overrightarrow{x},y,p(\overrightarrow{x},y))
\le \alpha_{K_2}(\max(\overrightarrow{x},y,p(\overrightarrow{x},y))) \le$$
$$\le \alpha_{K_2}(\max(\overrightarrow{x},y,\alpha^{y+1}_{K-2}(\max(\overrightarrow{x},y))) =$$
\center{(тут используется монотонность $\alpha$)}
$$= \alpha_{K_2}(\alpha^{y+1}_{K-2}(\max(\overrightarrow{x},y)))
%\le \alpha^{y+2}_{K-2}(\max(\overrightarrow{x},y)))\le$$
\le \alpha^{y+2}_{K-2}(\max(\overrightarrow{x},y+1))$$
\end{proof}

\begin{theorem}
Функция Аккермана не является примитивно-рекурсивной.
\end{theorem}

\begin{proof}
Пусть это не так, и функция $A(x,y)$ --- примитивно-рекурсивна.
Тогда рассмотрим $g(x) = A(x,x)$. Эта функция, очевидно, тоже является 
примитивно-рекурсивной. 

Однако, по предыдущей теореме, найдется такой $N$, что $g(x) \le A(N,x)$
при любом $x$. В том числе это верно и при $x=N+1$.
То есть, по предыдущей теореме, $g(N+1) = A(N+1,N+1) \le A(N,N+1)$, 
однако нам также известно, что $A(N+1,N+1) > A(N,N+1)$.
\end{proof}

\section{Арифметические отношения и функции. Их выразимость и представимость в формальной арифметике.}

\subsection{Выразимость и представимость}

Введем обозначение. Если в тексте вводится некоторая формула $\alpha(x_1, \dots x_n)$, то
по умолчанию считается, что эта формула имеет минимум $n$ свободных переменных, с именами
$x_1, \dots x_n$

Внутри же выражения запись $\alpha (y_1, \dots y_n)$ мы будем трактовать, как 
$\alpha [x_1 \coloneqq  y_1, ... x_n \coloneqq  y_n]$, при этом
мы подразумеваем, что $y_1, \dots y_n$ свободны для подстановки вместо $x_1, \dots x_n$ в $\alpha$.

Также, запись $B(x_1, \dots x_n) \equiv \alpha(x_1, \dots x_n)$ будет означать, что мы определяем
новую формулу с именем $B$ и $n$ свободными переменными $x_1, \dots x_n$. 
Данная формула должна восприниматься только как сокращение записи, макроподстановка.

\begin{definition}
Арифметическая функция --- функция $f: N^n \rightarrow N$.
Арифметическое отношение --- $n$-арное отношение, заданное на $N$.
\end{definition}

\begin{definition}
Арифметическое отношение $R$ называется выразимым (в формальной арифметике), если 
существует такая формула $\alpha (x_1, \dots x_n)$ с $n$ свободными переменными, 
что для любых натуральных чисел $k_1$ ... $k_n$
\begin{enumerate}
\item если $(k_1, \dots k_n) \in R$, то доказуемо $\alpha (\overline{k_1}, \dots \overline{k_n})$
\item если $(k_1, \dots k_n) \notin R $, то доказуемо $\neg \alpha (\overline{k_1}, \dots \overline{k_n})$.
\end{enumerate}
\end{definition}

Например, отношение $(<)$ является выразимым в арифметике:
Рассмотрим формулу $\alpha (a_1, a_2) = \exists b (\neg b = 0 \with a_1 + b = a_2)$.
В самом деле, если взять некоторые числа $k_1$ и $k_2$, такие, что $k_1 < k_2$, то найдется
такое положительное число $b$, что $k_1 + b = k_2$. Можно показать, что если подставить
$\overline{k_1}$ и $\overline{k_2}$ в $\alpha$, то формула будет доказуема. 

Наметим доказательство:
Тут должно быть два доказательства по индукции, сперва по $k_2$, потом по $k_1$.
Рассмотрим доказательство по индукции: пусть $k_1 = 0$, индукция по 2-му параметру:
Разберем доказательство базы при $k_2 = 1$. Тогда надо показать $\exists b (\neg b = 0 \with 0 + b = 1)$:

\begin{tabular}{lll}
(1) & $\neg 1 = 0 \with 0 + 1 = 1$ & Несложно показать\\
(2) & $(\neg 1 = 0 \with 0 + 1 = 1) \rightarrow \exists b (\neg b = 0 \with 0 + b = 1)$ & Cх. акс. для $\exists$\\
(3) & $\exists b (\neg b = 0 \with 0 + b = 1)$ & M.P. 1 и 2.
\end{tabular}

\begin{definition} Введем следующее сокращение записи:
пусть $\exists ! y \phi (y)$ означает $$\exists y \phi (y) \with \forall a \forall b (\phi(a) \with \phi(b) \rightarrow a=b)$$
Здесь $a$ и $b$ --- некоторые переменные, не входящие в формулу $\phi$ свободно.
\end{definition}

\begin{definition} Арифметическая функция $f$ от $n$ аргументов называется представимой в 
формальной арифметике, если существует такая формула $\alpha (x_1, \dots x_{n+1})$ с $n+1$ 
свободными пременными, что для любых натуральных чисел $k_1$ ... $k_{n+1}$
\begin{enumerate}
\item $f(k_1, \dots k_n) = k_{n+1}$ тогда и только тогда, когда доказуемо 
$\alpha (\overline{k_1}, \dots \overline{k_{n+1}})$.
\item Доказуемо $\exists ! b (\alpha (\overline{k_1}, \dots \overline{k_n}, b)$
\end{enumerate}
\end{definition} 

%$$f(x) = \left\{\begin{array}{rl}
%   \{x\}, & \mbox{если $P(x)$}\\
%   \emptyset, & \mbox{если $\neg P(x).$}
%\end{array}\right.

\subsection{Представимость рекурсивных функций}

\begin{theorem} Функции $Z$, $N$, $U^n_i$ являются представимыми. \end{theorem}
\begin{proof}
Наметим доказательство. Для этого приведем формулы, доказательство корректности этих 
формул оставим в виде упражнения.
\begin{itemize}
\item Примитив $Z$ представит формула $Z (a, b) \coloneqq  (a=a \with b=0)$.
\item Примитив $N$ представит формула $N (a, b) \coloneqq  (a' = b)$.
\item Примитив $U^n_i$ представит формула $U^n_i (a_1, ...a_n, b) = (a_1=a_1) \with ... \with (a_n=a_n) \with (b= a_i)$.
\end{itemize}
\end{proof}

\begin{theorem} Если функции $f$ и $g_1$, ... $g_m$ представимы, 
то функция $S\langle{}f,g_1,\dots g_m\rangle$ также представима. \end{theorem}
\begin{proof}Поскольку функции $f$ и $g_i$ представимы, то есть формулы $F$ и $G_1, \dots G_m$,
их представляющие. Тогда следующая формула представит $S\langle{}f,g_1,\dots g_m\rangle$: 
$$S (a_1, \dots a_n, b) \coloneqq  \exists b_1 \dots \exists b_m 
  (G_1 (a_1, \dots a_n, b_1) \with \dots \with G_m (a_1, \dots a_n, b_m) \with F (b_1, \dots b_m, b))$$
\end{proof}

\begin{definition}
$\beta$-функция Геделя - это функция $\beta (b,c,i) = b \% (1 + c \cdot (i + 1))$. Здесь операция (\%) означает
взятие остатка от целочисленного деления.
\end{definition}

\begin{lemma}Функция примитивно-рекурсивна, и при этом представима в арифметике 
формулой $B (b,c,i,d) \coloneqq  \exists q ((b = q \cdot (1 + c \cdot (i+1)) + d) \with (d < 1 + c \cdot (i+1)))$
\end{lemma}
\begin{proof}Упражнение.\end{proof}

\begin{theorem}Китайская теорема об остатках.
Если $u_1, \dots u_n$ --- попарно взаимно простые целые числа,
и $k_1, \dots k_n$ --- целые числа, такие, что $0 \le k_i < u_i$
при любом $i$, то найдется такое целое число $b$, что
$k_i = b \% u_i$ при любом $i$.
\end{theorem}

Доказательство этой теоремы по индукции несложно, но не входит в наш курс. 

\begin{lemma} 
Для любой конечной последовательности чисел $k_0$ ... $k_n$ можно подобрать
такие константы $b$ и $c$, что $\beta (b,c,i) = k_i$ для $0 \le i \le n$.
\end{lemma}

\begin{proof}
Возьмем число $c = \max(k_1,\dots k_n,n)!$. Рассмотрим числа $u_i = 1 + c \cdot (i+1)$. 

\begin{itemize}

\item Никакие числа $u_i$ и $u_j$ $(0 \le j < i \le n)$ не имеют общих делителей кроме 1.
Пусть это не так, и есть некоторый общий делитель $p$ (очевидно, мы можем предположить его
простоту --- разложив на множители, если он составной).
Тогда $p$ будет делить $u_i - u_j = c \cdot (i - j)$,
при этом $p$ не может делить $c$ --- иначе окажется, что $u_i = (1 + c \cdot (i+1))$ делится на $p$
и $c \cdot (i+1)$ делится на $p$. Значит, $p$ делит $i-j$, то есть все равно делит $c$, так как
$c$ --- факториал некоторого числа, не меньшего $n$, и при этом $i-j \le n$.

\item Каждое из чисел $k_i$ меньше, чем $u_i$: в самом деле, $k_i \le c < 1 + c \cdot (i+1) = u_i$.

Из полученных утверждений видно, что выполнены условия китайской 
теоремы об остатках.
Следовательно, найдется такое целое число $b$, для которого 
выполнено $k_i = b \% u_i$, то есть $k_i = \beta (b,c,i)$.
\end{itemize}
\end{proof}

\begin{theorem}Всякая рекурсивная функция представима в арифметике.\end{theorem}
\begin{proof}
Представимость первых четырех примитивов уже показана. Покажем представимость примитивной рекурсии и 
операции минимизации.

\emph{Примитивная рекурсия.} Пусть есть некоторый $R \langle{} f,g \rangle$. Соответственно, $f$ и $g$ уже 
представлены как некоторые формулы $F$ и $G$. Из определения $R\langle{}f,g\rangle$ мы знаем,
что для значения $R \langle{} f,g \rangle (x_1,...x_{n+1})$ должна существовать последовательность
$a_0 ... a_{x_{n+1}}$ результатов применения функций f и g --- значений на одно больше, чем 
итераций в цикле примитивной рекурсии,
а это количество определяется последним параметром функции $R \langle{}f,g\rangle$. При этом:

\begin{tabular}{l}
$a_0 = f(x_1, \dots x_n)$\\
$a_1 = g(x_1, \dots x_n,0,a_0)$\\
...\\
$a_{x_{n+1}} = g(x_1, \dots x_n, x_{n+1}-1,a_{x_{n+1}-1})$
\end{tabular}

Значит, по лемме, должны существовать такие числа $b$ и $c$, что
$\beta (b,c,i) = a_i$ для $0 \le i \le x_{n+1}$.

Приведенные рассуждения позволяют построить следующую формулу, представляющую $R\langle{}f,g\rangle (x_1, ... x_{n+1})$:

\begin{center}
$R(x_1, \dots x_{n+1}, a) \coloneqq  \exists b \exists c (\exists k (B (b,c,0,k) \with F (x_1,...x_n, k))$\\
       $\;\;\;\with B (b,c,x_{n+1},a)$\\
       $\;\;\;\with \forall k (k < x_{n+1} \rightarrow \exists d \exists e (B (b,c,k,d) \with B (b,c,k',e) \with
           G (x_1,..x_n,k,d,e))))$
\end{center}

\emph{Минимизация.} Рассмотрим конструкцию $\mu\langle{}f\rangle$. $f$ уже представлено 
как некоторая формула $F$. Тогда формула 
$M (x_1, \dots x_n,y) \coloneqq  F(x_1, \dots x_n,y,0) \with \forall z (z < y \rightarrow \neg F (x_1, \dots x_n,z,0))$
представит $\mu\langle{}f\rangle$.
\end{proof}

\begin{theorem}Всякое рекурсивное арифметическое отношение выразимо в формальной арифметике.\end{theorem}
\begin{proof}Рассмотрим характеристическую функцию $C_R$ некоторого рекурсивного отношения $R$. Данная функция 
(как любая другая рекурсивная функция) представима в формальной арифметике --- пусть
ее представляет формула $\rho(x_1, \dots x_n, y)$. Тогда 
формула $\rho (x_1, \dots x_n, \overline{1})$ выразит соответствующее отношение.

В самом деле, если $(k_1,\dots k_n) \in R$ для некоторых $k_i$, то
$\vdash \rho(\overline{k_1}, \dots \overline{k_n}, \overline{1})$.

Обратно, пусть $(k_1,\dots k_n) \notin R$. Из представимости $C_R$ известно
$$\vdash \exists ! y\rho(\overline{k_1}, \dots \overline{k_n}, y)$$
то есть, в частности,
$$\vdash \forall y_1 \forall y_2 (\rho(\overline{k_1}, \dots \overline{k_n}, y_1) \with \rho(\overline{k_1}, \dots \overline{k_n}, y_2) \rightarrow y_1 = y_2)$$
что можно преобразовать в
$$\vdash \rho(\overline{k_1}, \dots \overline{k_n}, 0) \rightarrow (\rho(\overline{k_1}, \dots \overline{k_n}, \overline{1}) \rightarrow \overline{1} = 0)$$
Поскольку $\vdash \rho(\overline{k_1}, \dots \overline{k_n}, 0)$ доказуемо вследствие представимости $C_R$,
то мы можем далее вывести
$$\vdash \rho(\overline{k_1}, \dots \overline{k_n}, \overline{1}) \rightarrow \overline{1} = 0$$
и, заметив, что $\vdash\neg \overline{1} = 0$, в итоге покажем требуемое для выразимости
$$\vdash \neg\rho(\overline{k_1}, \dots \overline{k_n}, \overline{1})$$
\end{proof}
