\documentclass[12pt,a4paper,oneside]{book}
\usepackage[utf8]{inputenc}
\usepackage[english,russian]{babel}
\begin{document}

\begin{center}
\begin{Large}Программа курса <<Математическая логика>>\end{Large}\\
ИТМО, группы 2537-2539, осень 2012 г.
\end{center}

\begin{enumerate}
\item Исчисление высказываний. Общезначимость, доказуемость и выводимость.
\item Теорема о дедукции для исчисления высказываний.
\item Теорема о полноте исчисления высказываний.
\item Интуиционистское исчисление высказываний. Топологическая интерпретация исчисления, модели Крипке.
\item Исчисление предикатов. Общезначимость и выводимость. 
\item Теорема о дедукции в исчислении предикатов. Корректность исчисления предикатов.
\item Теорема о полноте исчисления предикатов.
\item Теория 1-го порядка, модели. Аксиоматика Пеано. Формальная арифметика. 
\item Рекурсивные функции и отношения. Базовые операции (сложение, умножение, 
вычитание, деление).
\item Выразимость отношений и преставимость функций в формальной арифметике.
Представимость примитивов $Z$, $N$, $U$ и $S$.
\item Бета-функция Гёделя. Представимость примитивов $R$ и $\mu$ в формальной арифметике. Представимость
рекурсивных функций.
\item Гёделева нумерация. Выводимость и рекурсивные функции.
\item Непротиворечивость и $\omega$-непротиворечивость. 1я и 2я теоремы 
Гёделя о неполноте арифметики.
\item Теория множеств. Аксиоматика Цермело-Френкеля.
\item Ординальные и кардинальные числа, мощность множества.
\item Теорема Лёвенгейма-Сколема. Парадокс Сколема.
\end{enumerate}

\end{document}