\section{Литература}
Возможно, в ходе подготовки вам потребуются дополнительные источники.
В таком случае рекомендуется использовать следующие книги:

\begin{itemize}
\item Классическое исчисление высказываний и предикатов:

С. Клини. Математическая логика --- М.: Изд-во <<Мир>>, 1973

Н.К. Верещагин, А. Шень, Лекции по математической логике и теории алгоритмов, Языки и исчисления --- МЦНМО, 2002.
Также доступно по ссылке\\
\s{http://www.mccme.ru/free-books/shen/shen-logic-part2.pdf}

\item Интуиционистская логика: 

Н.К. Верещагин, А. Шень, Лекции по математической логике и теории алгоритмов, Языки и исчисления --- МЦНМО, 2002.
Также доступно по ссылке\\
\s{http://www.mccme.ru/free-books/shen/shen-logic-part2.pdf}

\item Теорема Геделя о неполноте арифметики: 

Э. Мендельсон. Введение в математическую логику --- М.: Изд-во <<Наука>>, 1971.

\item Теория множеств: 

А.А. Френкель, И. Бар-Хиллел. Основания теории множеств --- М.: Изд-во <<Мир>>, 1966.

П.Дж. Коэн. Теория множеств и континуум-гипотеза --- М.: Изд-во <<Мир>>, 1969.

\end{itemize}
