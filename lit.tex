\section{Литература}

\begin{thebibliography}{99}

%Классическое исчисление высказываний и предикатов:

\bibitem{klinilogic} С. Клини. Математическая логика --- М.: Изд-во <<Мир>>, 1973
%\bibitem{shen2} Н.К. Верещагин, А. Шень, Лекции по математической логике и теории алгоритмов, Языки и исчисления --- МЦНМО, 2002.
% Также доступно по ссылке \s{http://www.mccme.ru/free-books/shen/shen-logic-part2.pdf}

%Интуиционистская логика: 

\bibitem{shen2} Н.К. Верещагин, А. Шень, Лекции по математической логике и теории алгоритмов, Языки и исчисления --- МЦНМО, 2002.
 Также доступно по ссылке \s{http://www.mccme.ru/free-books/shen/shen-logic-part2.pdf}

%Теорема Геделя о неполноте арифметики: 

\bibitem{shen3} Н.К. Верещагин, А. Шень, Лекции по математической логике и теории алгоритмов, Вычислимые функции --- МЦНМО, 2002.
 Также доступно по ссылке \s{http://www.mccme.ru/free-books/shen/shen-logic-part3.pdf}
\bibitem{mendel} Э. Мендельсон. Введение в математическую логику --- М.: Изд-во <<Наука>>, 1971.
\bibitem{klinimeta} С. Клини. Введение в метаматематику --- М.: Изд-во <<Иностранная литература>>, 1957.
\bibitem{kikuchi} Makoto Kikuchi. Kolmogorov complexity and the second incompleteness theorem. Arch. Math. Logic 36: 437-443 (1997)
\bibitem{kritchman} Shira Kritchman, Ran Raz. The Surprise Examination Paradox and the Second Incompleteness Theorem. Notices of the AMS volume 57(11): 1454-1458 (2010)

%Теория множеств: 

\bibitem{frenkel} А.А. Френкель, И. Бар-Хиллел. Основания теории множеств --- М.: Изд-во <<Мир>>, 1966.
\bibitem{koen} П.Дж. Коэн. Теория множеств и континуум-гипотеза --- М.: Изд-во <<Мир>>, 1969.

\end{thebibliography}